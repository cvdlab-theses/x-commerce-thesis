\section{NodeJS}
\label{sec:nodejs}
This section provides an overview of NodeJs.
Node.js is an open source, cross-platform runtime environment for server- side and networking applications. Node.js applications are written in JavaScript and can be run within the Node.js runtime on OS X, Microsoft Windows, Linux, FreeBSD, NonStop,IBM AIX, IBM System z and IBM i. Its work is hosted and supported by the Node.js Foundation,a Collaborative Project at Linux Foundation.
\newline
Node.js provides an event-driven architecture and a non-blocking I/O API that optimizes an application’s throughput and scalability. These tech- nologies are commonly used for real-time web applications.
\newline
Node.js uses the Google V8 JavaScript engine to execute code, and a large percentage of the basic modules are written in JavaScript. Node.js contains a built-in library to allow applications to act as a Web server without software such as Apache HTTP Server, Nginx or IIS.
\newline
Node.js allows the creation of web servers and networking tools, using JavaScript and a collection of “modules” that handle various core functional- ity. Modules handle file system I/O, networking (HTTP, TCP, UDP, DNS, or TLS/SSL), binary data (buffers), cryptography functions, data streams , and other core functions. Node’s modules have a simple and elegant API, reducing the complexity of writing server applications.
Frameworks can be used to accelerate the development of applications, and common frameworks are Express.js, Socket.IO and Connect. Node.js applications can run on Microsoft Windows, Unix, NonStop and Mac OS X servers. Node.js applications can alternatively be written with CoffeeScript (an alternative form of JavaScript), Dart or Microsoft TypeScript (strongly typed forms of JavaScript), or any language that can compile to JavaScript.
Node.js is primarily used to build network programs such as web servers, making it similar to PHP and Python. The biggest difference between PHP and Node.js is that PHP is a blocking language (commands execute only after the previous command has completed), while Node.js is a non-blocking language (commands execute in parallel, and use callbacks to signal completion).
\newline
Node.js brings event-driven programming to web servers, enabling development of fast web servers in JavaScript. Developers can create highly scalable servers without using threading, by using a simplified model of event- driven programming that uses callbacks to signal the completion of a task. Node.js was created because concurrency is difficult in many server-side programming languages, and often leads to poor performance. Node.js connects the ease of a scripting language (JavaScript) with the power of Unix network programming.
