{\color{red} -----------------------corretto il quarto capitolo fino a qui----------------------}
\section{Execute tasks to retry payment}
\label{sec:tasks_to_retry_payment}
The payment of an order can have different outcomes in particular:
\begin{itemize}
\item It can fail for the following reasons:
\begin{itemize}
\item error in data entry of credit card;
\item for network problems;
\item insufficient credit;
\item unknown reasons;
\end{itemize}
\item transaction successfully completed;
\end{itemize}
In each of these cases the customer e-commerce must be notified of the outcome of the transaction.
\newline
In the case in which the customer inserts the data of the credit card incorrect then it is immediately alerted.
\newline
If the data on the card are correct but there are other problems for which the first transaction fails to complete for reasons listed above, Braintree then returns a reply containing identifier of the failed transaction.
\newline
This transaction ID (and other information) is stored in the DB to retry payment in particular is used to store the template \emph{task} this information.
In particular, if a payment transaction fails then the server x-commerce creates a instance of the task model with the following data:
\begin{lstlisting}[language=javascript]
  var get_task_braintree = function (data) {
    var date_now = moment().format().split('+')[0] + 'Z';
    var task = {
      data: {
        order_id: data.order.id,
        transaction_id: data.payment_status.transaction.id,
        customer_id: data.customer.id,
        payment_system: 'braintree'
      },
      handler: 'retry_payment',
      created_at: date_now,
      priority: 'medium',
      last_retry_at: date_now,
      retry_count: 1,
      done: false
    };
    return task;
  };
\end{lstlisting}
When starts of x-commerce starts a cron that appropriate and regular intervals starts and checks if there are tasks to be performed.
A Cron is a time-based job scheduler in Unix-like computer operating systems.
The function of this scheduler is to verify the presence of tasks to be performed in particular:
\begin{itemize}
\item if there are no tasks to be performed then the crohn falls asleep;
\item if there are any cron task then it takes all tasks and we select a task to be carried out with a policy implemented in the following function:
\begin{lstlisting}[language=javascript]
  var get_next_task = function (tasks) {
    var test = false;
    var task = null;
    for(var i=0; i < tasks.length && !test; i++) {
      var last_retry_at = new Date(tasks[i].last_retry_at)
      var date_now = new Date(moment().format().split('+')[0] + 'Z');
      var minutes_past = (date_now - last_retry_at)/1000/60;
      if (minutes_past > Math.pow(tasks[i].retry_count, 4.09)) {
        task = tasks[i];
        test = true;
      }
    }
    return task;
  };
\end{lstlisting}
Selecting a task to perform depends on two main factors:\begin{itemize}
\item the number of attempts to retry the task;
\item the time since the last time the task was executed
\end{itemize}
In particular, the probability that a task is selected decreases as the number of attempts made for the task.
For example:
\begin{enumerate}
\item if a task with the retry\_count = 0 => \(0^{4,09} = 1\). Then this task is selected to run if it is spend at least one minute;
\item if a task with the retry\_count = 1 => \(1^{4,09} = 4,09\). Then this task is selected to run if they spent at least 4,09 minutes;
\item if a task with the retry\_count = 2 => \(2^{4,09} = 17,02\). Then this task is selected to run if they spent at least 17.02 minutes;
\item if a task with the retry\_count = 3 => \(3^{4,09} = 89,41\). Then this task is selected to run if they spent at least 89,41 minutes;
\item if a task with the retry\_count = 4 => \(4^{4,09} = 290,01\). Then this task is selected to run if they spent at least 290,01 minutes;
\end{enumerate}
As you can see, a new task is now selected to be tried again. Instead other tasks that continue to fail will gradually discarded.
\newline
This idea to try to make payment by executing the task is very important namely when the customer has entered the data of the credit card and did checkout then it is the responsibility of the administrator of the platform, the transaction was completed successfully without requiring the customer to try again.
\newline
This is appropriate because the customer could change his mind if you continue to ask him to refuse the payment. So it is advisable that the customer confirms the payment once only and all that needs to be managed at the server side.
\end{itemize}