\section{HTML5}
\label{sec:html5}
This section provides an overview of HTML5.
HTML5 is the latest version of Hypertext Markup Language, the code
\begin{figure}[htb]
 \centering
 \includegraphics[width=1.0\linewidth]{images/chapter2/tecnologie.jpeg}\hfill
 \caption[Server and client sides enabling  technologies]{Server and client sides enabling  technologies}
 \label{fig:fourV}
\end{figure}
that describes web pages. There are actually three kinds of code: HTML, which provides the structure; Cascading Style Sheets (CSS), which take care of presentation; and JavaScript, which makes things happen.
\newline
HTML5 has been designed to deliver almost everything it is possible to do online without requiring additional software such as browser plugins. It does everything, from animation to apps, music to movies, and can also be used to build complicated applications that run in browsers.
\newline
Moreover, HTML5 isn’t proprietary, so it is completely free. It’s also a cross-platform standard, which means it doesn’t care whether the device is a tablet or a smartphone, a netbook, notebook or ultrabook or a Smart TV: if the browser supports HTML5, it should work  flawlessly.
\newline
While some features of HTML5 are often compared to Adobe Flash, the two technologies are very different. Both include features for playing audio and video within web pages, and for using Scalable Vector Graphics. HTML5, on its own, cannot be used for animation or interactivity, it must be supplemented with CSS3 or JavaScript. There are many Flash capabilities  that have no direct counterpart in HTML5. See Comparison of HTML5 and Flash.
\begin{figure}[htb]
 \centering
 \includegraphics[width=1.0\linewidth]{images/chapter2/Html5_responsive.png}\hfill
 \caption[Html5 Responsive]{Html5 Responsive}
 \label{fig:fourV}
\end{figure}
Although HTML5 has been well known among web developers for years, its interactive capabilities became a topic of mainstream media around April 2010, after Apple Inc’s then-CEO Steve Jobs issued a public letter entitled “Thoughts on Flash”  where he concluded that “Flash is no longer necessary to watch video or consume any kind of web content” and that “new open standards created in the mobile era, such as HTML5, will win”.This sparked a debate in web development circles where some suggested that while HTML5 provides enhanced functionality, developers must consider the varying browser support of the different parts of the standard as well as other functionality differences between HTML5 and Flash. In early November 2011, Adobe announced that it would discontinue development of Flash for mobile devices and reorient its efforts in developing tools using HTML5.
